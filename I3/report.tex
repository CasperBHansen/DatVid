\documentclass[a4paper]{article}


%========== PACKAGES ==========%

% \usepackage{a4wide} % save a few rainforests
\usepackage{amsmath,amssymb} % mathematical notation
\usepackage[danish]{babel}
\usepackage{color} % pretty colors
\usepackage{fancyhdr} % make things look fancy
\usepackage{float} % put things exactly where i tell you
\usepackage{listings} % code listings
\usepackage[utf8]{inputenc} % can i has UTF-8 plox


%========== DEFINITIONS ==========%

\definecolor{c_comment}{rgb} 	{0.38, 0.62, 0.38}
\definecolor{c_keyword}{rgb} 	{0.10, 0.10, 0.81}
\definecolor{c_identifier}{rgb} {0.00, 0.00, 0.00}
\definecolor{c_string}{rgb} 	{0.50, 0.50, 0.50}


%========== SETTINGS ==========%

\lstset
{
	numbers=left,
	frame=single,
	basicstyle=\footnotesize\ttfamily,
	tabsize=4,
	% colors
	commentstyle=\color{c_comment},
	keywordstyle=\color{c_keyword},
	identifierstyle=\color{c_identifier},
	stringstyle=\color{c_string}
}


%========== META DATA ==========%

\title
{
	{\large Theory of Science for Computer Science}\\
	Individual Assignment 3
}

\author
{
	Casper B. Hansen\\
	University of Copenhagen\\
	Department of Computer Science\\
	{\tt fvx507@alumni.ku.dk}
}

\date{\today}


%========== DOCUMENT ==========%

\begin{document}

\clearpage
\maketitle
\thispagestyle{empty}

\newpage
\section{Redegørelse}

\subsection{Kan Peter Hansen forlange at få fjernet fotografiet fra
hjemmesiden?}
I det pågældende tilfælde forpligter databehandleren Jens Nørregård sig til,
at oplyse ophavsmanden fotograf Peter Hansen om brugen af materialet, som er
indbefattet af loven om ophavsret\cite{ophav} jvf. §70, stk. 1. I denne
kontekst har ophavsrettens indehaver Peter Hansen jvf. §38, stk. 1--7, krav på
vederlag for sit værk. Hvis dette krav ikke mødes jvf. §76, stk. 1 i kraft, og
medfører således straf for databehandleren Peter Hansen.

I det store hele, så er det §2 stk. 1--4 der krænkes, som indleder\cite{ophav}
``Ophavsretten medfører, med de i denne lov angivne indskrænkninger, eneret
til at råde over værket ved at fremstille eksemplarer af det og ved at gøre
det tilgængeligt for almenheden i oprindelig eller ændret skikkelse'', og det
er her tilfældet, at ophavsmanden Peter Hansen's rettigheder er jvf. nævnte
paragraf, blevet krænket af databehandleren Jens Nørregård.

Som følge af overtrædelse heraf, vil det være op til Peter Hansen om hvorvidt
han, som ophavsrettighedens indehaver, ønsker at afvikle ved §38, stk. 1--7 og
modtage vederlag for værkets anvendelse, eller om han ønsker dette fjernet.

\subsection{Giver undersiden ``Syng med fra højskolesangbogen'' anledning til
ophavsretslige problemer?}
Det kan jeg ikke svare på med sikkerhed\footnote{Jeg forsøgte at undersøge
om nogen af de daværende ophavsmænd til hhv. tekst og melodi har levende
afkom}, og årsagen er, at jvf. §63, stk. 1--4 i bekendtgørelsen af lov om
ophavsret\cite{ophav}, som indleder ``Ophavsretten til et værk varer, indtil
70 år er forløbet efter ophavsmandens dødsår eller for de i §6 omhandlede
værker efter længstlevendes dødsår''.

I og med, at jeg fejlede i at producere et konkret svar på, om nogle af de
daværende ophavsmænd havde ladet ophavsretten gå i arv, kan jeg ikke give et
konkret svar. Jeg kan dog påpege, at hvis det er tilfældet, så vil der et
ophavsretligt problem med undersiden. Hvis ikke, så er det fuldstændig lovligt
og bør ikke give anledning til nogen ophavsretslige problemer.

\newpage
\subsection{Må Bøgely forære eksemplarer af Lissi Petersen's træningsprogram
til eleverne? Og må Bøgely fremstille ekstra kopier af programmet til
uddeling, hvis der ikke er nok til alle deltagerne? Har Ursula krænket Lissi
Petersen's ophavsret?}
De 50 kopier, som købt på lovlig vis ser jeg ikke nogen lov imod om Bøgely må
{\it forære} dem væk, medmindre det indbefatter {\it videresalg}, hvilket der
ikke oplyses om. Derimod må Bøgely ikke fremstille ekstra kopier uden samtykke
fra ophavsmanden Lissi Petersen jvf. loven om ophavsret\cite{ophav} §12, hvor
det som det fremgår af det pågældende tilfælde hverken er til privat brug og
samtidig også er ved erhvervsøjemed.

Såfremt Ursula's datter Lizette hører under samme husstand som moderen, er det
fuldt ud lovligt\cite{ophav}, at ``fremstille enkelte eksemplarer i digital
form af andre værker end edb-programmer og databaser, [hvis] det udelukkende
sker til personlig brug for fremstilleren eller dennes husstand.'', jvf. §12,
og derfor har Ursula {\it ikke} krænket Lissi Petersen's ophavsret.

\subsection{Har Jens Nørregård skabt ophavsretslige problemer med
databasesystemet?}
Elevdatabasen er udviklet på kilde, som Jens Nørregård har en gyldig licens
til at videreudvikle på og formodentlig implementere i kommerciel forbindelse,
derfor ser jeg ingen anledning til ophavsretslige problemer her. Ligeledes må
man formode, at siden Nina videregav hendes kildekode til Jens, har hun i
samme forbindelse givet sin samtykke til, at Jens måtte bruge denne. Og siden
Jørgen Olsen er medarbejder i ``Nørregård Consulting'', må man antage at den
supplerende kode til open source koden var udarbejdet under
ansættelseskontraktens forhold, hvor Jørgen Olsen giver afkald på denne
ophavsret, og overdrager den til ``Nørregård Consulting'', samt open source er
per definition åbent til fri afbenyttelse, også i kommerciel henseende. Ved
videresalget af licensen til databaseprogrammet til højskolen forekommer der
heller ingen ophavsretsmæssige problemer.

Det der bør påpeges med henblik på databaserne er, at tidligere elever,
lærere og oplægsholderes personoplysninger behandles, men under formodningen
om, at disse har givet samtykke til at skolen må beholde disse data, er det
blot redelegering af dataene fra, angiveligvis papir arkiver, over i et
databasesystem. Hertil har Jens Nørregård selvfølgelig, som databehandler
ansvaret for at disse oplysninger bliver behandlet på en sikker måde. Det
samme gælder for behandlingen af leverandøroplysningerne, som også behandles.

\newpage
\section{Vurdering}
Bøgely bør i første ombæring kontakte en advokat, hvis medvirken kunne
afhjælpe alle de påpegede (omend ikke flere) ophavsretslige problemer, da en
advokat er langt bedre kvalificeret til at identificere og vurdere disse
problemer.

Mit personlige bud på løsning heraf kunne være, at bla. indgå en aftale med
Lissi Petersen om licensering af det anvendte materiale på en sådan mannér, at
det tillader højskolen at udlån af materialet. Hvis den enkelte elev ønsker,
at erhverve sig en kopi bør det ske udenom højskolens regi. Bed om tilladelse
til at benytte, under forhandling med ophavsmanden Peter Hansen, det
fotografiske materiale. Hvis ude af stand til at indgå en aftale herom bør
materialet fjernes, for at undgå straf ved brud på lov herom. Undersøg, om de
daværende ophavsmænd til materialet anvendt på undersiden ``Syng med fra
højskolesangbogen'' har arvinger i live, som da vil have den nuværende
ophavsret, eller om denne er ugyldig. Hvis det er tilfældet, anmod derved om
tilladelse til brugen af materialet og indgå vederlagsaftale, eller fjern
materialet.

\section{Perspektivering}
Essensen af de ophavsretslige regler er, at beskytte ophavsmandens rettigheder
omkring det pågældende materiale. Det er i og for sig vigtigt at beskytte
disse rettigheder, men omfanget heraf gør ofte datalogens opgave væsentligt
mere vanskelig. Det bør i første ombæring ikke være det en datalog skal
beskæftige sig med at undersøge, men hvis nødvendigt redelegeres videre til en
sagkyndig på området, altså en advokat.

Der er dog et endnu større problem for datalogens virke, og det er når man er
overbevist om, at dele af ens kode er fuldkommen original, uvidende om, at den
algoritme man har implementeret er patenteret. Dette giver anledning til mange
unødige problemer, som datalogen skal tage sig af, og endda kan blive ruineret
på baggrund af. Problemet opstår i, at det nu er muligt at patentere og derved
opnå ophavsretten for abstrakte koncepter, som har en bred anvendelighed i
datalogisk sammenhæng, hvilket gør at selvom en patent er givet til et bestemt
formål da det blev oprettet, så gør den abstrakte natur af metodens
beskrivelse, at den kan fortolkes og tilpasses andre anvendelses områder ---
hvilket naturligvis kommer som direkte følge af datalogiens metodik; at
arbejde i forskellige abstraktionslag og anvende abstrakte koncepter, hvilket
selvfølgelig giver anledning til at beskrive metoderne i abstrakte termer, og
derved også patent beskrivelser.

I denne sammenhæng kan jeg kun pege på politikerne og bede om deres forståelse
af, at abstrakte metoder ikke bør kunne patenteres, dog med det forbehold, at
software, som en helhed (dvs. konkrete produkter) skal bibeholde denne
rettighed. I sidste ende, hvis dette fortsætter, så ved ingen dataloger om de
må skrive kode på en bestemt måde mere uden det har været igennem en advokat.

Afslutningsvis, så vil jeg fremhæve en dokumentar om netop dette, som hedder
``Patent Absurdity'', der er frit tilgængelig fra ophavsmændenes egen
hjemmeside\footnote{Patent Absurdity ---
{\tt http://www.patentabsurdity.com/}}, hvor læseren kan få yderligere
information herom.

\newpage
\begin{thebibliography}{9}

\bibitem{blume}
	Peter Blume, \\
	\emph{Den Digital Revolution --- fortællinger fra datalogiens verden,
	Krav til informationssikkerhed}. \\
	Datalogisk Institut ved Københavns Universitet, 2010

\bibitem{ophav}
	Kulturministeriet, \\
	\emph{Bekendtgørelse af lov om ophavsret}. \\
	Ophavsretsloven, Offentliggørelsesdato 09/03/2010

\end{thebibliography}

\end{document}

