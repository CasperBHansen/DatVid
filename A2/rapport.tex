\documentclass[11pt,a4paper]{article}

\usepackage{a4wide}
\usepackage[utf8]{inputenc}

\title{A Discussion on The Scientific Method
       \\\hfill\\
       \large{In response to Wired’s article on the possibility of the obsolescence thereof}}
\author
{
	Adam Honoré\\
	{\tt qgf142@alumni.ku.dk}
	\and
	Casper B. Hansen\\
	{\tt fvx507@alumni.ku.dk}
	\and
	Maya Saietz\\
	{\tt xdk201@alumni.ku.dk}
}

\begin{document}
\maketitle

\section{The Scientific Method}
Scientific method, as explained in\cite{okasha02} is defined as the process of doing experiments and observing their outcome, leading to the construction of a theory.

\paragraph{Hypothesis}
The first phase of anything in science is to form one or more falsifiable hypothesis -- that is, an idea about what the answer \emph{might} be, which can be tested.

\paragraph{Experiment}
In the experimentation phase we set up and execute experiments based on our hypothesis, with the goal of either falsifying or confirming it -- although the experimental outcome may be something completely different than our original ideas and thoughts. In that case, we form one or more new hypothesis.

\paragraph{Observation}
In the phase of observation we observe and note the results of our experiments and their outcome -- expected or not.

\paragraph{Theory}
At some point -- hopefully -- we end up with a hypothesis that we can’t falsify. If we, in trying to falsify the initial hypothesis fail to do so, we are forced to promote it to a theory. Note that a theory in science is something very different from a ``theory'' in casual speech -- in science, when we accept something as a theory, we can ascertain its validity, hence making it useful for further development.

\section{Discussion}
Chris Anderson makes some quite bold statements about science in relation to recent developments with regard to methodology of approach in analyzing data. He came to these conclusions in light of what he refers to as \emph{The Petabyte Age}, which signifies the abundance of data available, as well as the computing power necessary to process such large datasets.

According to Chris Anderson the scientific method, as discussed above, has become obsolete by reasoning in differences of methodology, where he believes that the use of statistical analysis is a suitable substitute of the scientific method. Despite his impressive background in the study of science we are inclined to disagree with his conclusions for several reasons; statistical data does not provide any guarantee as to its findings (e.g. it is well-documented that the principle of the wisdom of the crowd works, but its conclusions are always just estimates). Although a strong indicator, we cannot produce any definite conclusion purely based on an estimate formed from statistical data.

While Anderson believes that conceiving models from sets of data to form theoretical conclusions is an obsolete method, we then ask what is to become of the arbitrary conclusions, as they seem to have no apparent reasoning backing it up. How does one go about using such a finding, unless strictly within the domain of which it was found? Can we positively put \emph{that} it is to any good use without knowing \emph{why} it is? That is indeed what is characteristic of general theories -- that they can be applied anywhere.

Even if we were convinced that correlation, given enough data, does imply causation, we still have some disagreements with Anderson's idea that this is all we need to know. Simply figuring out causality between phenomena is not actually useful -- in order to use our discoveries to affect the world around us, we need a model that can predict the results of our actions. For example, knowing that there exists some previously unknown species\cite{anderson08} is all very well, but without an underlying model of how DNA works, it won’t help us create new species ourselves.

A model also serves the purpose of telling us \emph{which} pieces of data are relevant for a certain goal -- for example, if we want to build something that flies, a natural first step would be to look at how things that can already fly do it. We’ll soon find a correlation between things that have wings and things that fly. We’ll also find a correlation between things that have feathers and things that fly, and while the feathers \emph{are} relevant to how birds fly, that doesn’t mean we can't apply some of the wing-related things to featherless airplanes.

Or, more generally, science is about explaining the world around us. If you leave the explanation out of the process, how would you teach newly found ``knowledge'' to students, if all you have is a large body of statistical data? How would you convey such findings?

\section{Natural Deduction}
Putting into question the trusted methodology of the scientific community one cannot help but think of all the areas in which its use is essential. Relating the views of Chris Anderson to a field heavily based on the scientific method, namely the field of mathematical logic, we would like to put his presumptuous views to the test.

In deriving a proof in mathematical logic we have to state our premises within the constraints of a given model and our assumptions as we make them along the way. Having no premises simply means we are free of any constraints. Introducing premises, however, such as large data sets, does indeed influence what we can conclude and hence the model around which we build our proof. From these premises, if any, we can derive many intermediate conclusions, before reaching the desired conclusion of a given hypothesis -- that is, a definite answer as to the question of whether or not it holds. Along the way we may or may not state some assumptions about the model, which in accordance with the description of the scientific method, as described above, could be hypotheses which have yet to be proven. Not only are these intermediate steps of the proof valuable in terms of their theoretical explanation of the conclusion, but also allows us to form theories out of yet-to-be-proven hypotheses by making such assumptions.

Substituting our rigorous scientific methods with the lazy attitude, such as the one presented by Anderson, does not prove satisfactory in any scientific area, as its conclusions are worthless outside the scope in which it was found.

\section{Concluding Remarks}
Science is, and has always been, about understanding the world around us. Leaving out the explanation and letting statistical data dictate our conclusions, and hence beliefs, is in direct conflict with what science is truly about. What Anderson is advocating is faith, not science. The very nature of the scientific method enables us to think not only factually about tangible evidence and the possible theories to be derived from such observations, but also allows us to be creative in the process, and ask questions that may or may not yield a theory.

We ask Anderson; ought we not to ask questions? Shouldn’t we exercise the curiousness which makes us human? We certainly think so.

\newpage
\begin{thebibliography}{9}
    \bibitem{okasha02}
        Samir Okasha,
        Philosophy of Science: A Very Short Introduction,
        Oxford University Press,
        2002

    \bibitem{anderson08}
        Chris Anderson,
        The End of Theory: The Data Deluge Makes the Scientific Method Obsolete,
        http://www.wired.com/science/discoveries/magazine/16-07/pb\_theory,
        2008

    \bibitem{translate}
        Google Translate,
        http://translate.google.com/about/
\end{thebibliography}

\end{document}
