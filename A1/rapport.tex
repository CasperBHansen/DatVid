\documentclass{article}

\title{A comparison of “Computer Science Curriculum 2013” and Peter Naurs description of the field of datalogy}
\author{Adam Honoré, Casper B. Hansen and Maya Saietz}

\begin{document}
\maketitle

\noindent Computer science is a relatively new field, and so it is still not entirely clear which topics should be taught in a
Computer Science education. The University of Copenhagen adheres to the Copenhagen Tradition, which was largely defined
by Peter Naur. Here, we present a comparison between the Copenhagen Tradition, and the more conventional Computer
Science Curriculum 2013, which was formulated by the ACM and IEEE.

\section{Datalogy in contrast to Computer science}
There should be a clear distinction between the terms “Computer Science” and “Datalogy”. As the article[2] puts it, “ACM
Curriculum seems to make an attempt to cover the field in an almost encyclopedic manner, apparently making sure to
mention all the current techniques, languages and practices, however briefly, while the datalogy course strives to
emphasize the underlying ideas and principles”.

Naur describes the field of datalogy as a tool like mathematics and linguistics. None of these lend themselves to any
specific purpose, but rather preserves the generality and applicability for any number of different problem classes. At
the same time the ACM is rather focused on computer science as a tool to solve problems specific to the field of
computer science, and not to solve problems in other fields.

While the ACM curriculum has been changed since this was written, and now admits that it is simply not possible to cover
everything, it still puts a lot of topics in the category of “core” topics. In contrast, Naur argues for a focus on
data, data representation and data processing[1].

The article[2] makes it very clear that from the very early stages of its conception, education in computer science has
been lacking important aspects of the discipline, such as project work and practical work. Recognizing this neglect,
Peter Naur formulates, in collaboration with several people, what is known today as datalogy.

Peter Naur defines the central concepts of datalogy as data, data representations and data processing, while the ACM
puts many more topics in the Core category. Although, in later years, particularly with the advent of CS2013, the ACM
has acknowledged more and more, many of the important points Peter Naur had early on.

\section{Computer Science Curriculum 2013}
The curricula guidelines presented [3] by the ACM reveals that the ACM/IEEE are only just beginning to grasp the points
made by Peter Naur earlier in the history of the development of computer science as an education. E.g. it had previously
been the conviction at ACM that the focus of the curriculum should be on all recent developments, current techniques and
give the students a full understanding of every topic [2] within computer science. Bullet 6 states that “While the range
of relevant topics has expanded, the size of undergraduate education has not. Thus, CS2013 must carefully choose among
topics and recommend the essential elements.”, which is closer to what Naur said [2] early on; that in computer science
we should focus our teaching efforts on emphasizing “the underlying ideas and principles, while omitting many particular
instances of the various notions”. Likewise, bullet 7 of [3] is in direct agreement with Naurs early views [2], that
“Comprehensive project activity is an integral part of the curriculum, thus presenting theory as an aspect of realistic
solutions known primarily through actual experience”.

\section{Similarities}
Even though the ACM/IEEE and Peter Naur disagreed on a great many things, there are of course also a lot of which they
did, and still do agree on. For instance, the applicability of the field is typically combined with other fields, as
described in both [2] and [3].

Naur’s main point is that it is important to focus on some core concepts, rather than trying to teach all the specific
techniques. The ACM also says that it is impossible to cover the entire field, and that it is necessary to select
carefully the topics that will best prepare the students to adapt in an ever-changing field. It is also said in the ACM
that students are committing to a life-long process of learning as the field develops, which is another way of saying
that they can’t learn everything from courses.

The ACM acknowledges the variety and growing amount of topics and their independence, and efforts are placed on topics
more central to the core of computer science, while maintaining the flexibility of specializing students in various
non-general topics. By this it is believed that by understanding the underlying principles, students will be able to
understand and learn any technique and thus not need it addressed in the curriculum. From the perspective of ACM this
takes the form of drawing connections between the individual courses, understanding recurring themes, while datalogy
addresses this by directly teaching those principles with the intention of expanding upon this knowledge at a later
course, or simply leaving it up to the students to apply the acquired knowledge during their life-long learning.

The ACM says that social and communicative skills should be practiced, and Naur agrees with this - students should learn
to present material in various formats (text, slideshows, formal documentation).
The ACM agrees with Naur on the point of learning problem solving skills. Although Naur mentions it in relation to
project work where students solve larger problems together, it is written more specifically by the ACM as a skillset
that students are required to learn.

\section{Discrepancies}
Despite the fact that ACM/IEEE have taken on a radically different point of view on many different aspects since the
early conception of the ideal computer science curriculum, to such a degree, that many things almost exactly reflects
that which Peter Naur pointed out early on, some differences still exists.

The ACM still has a wide array of topics that are considered core material, and argues that a student must, to some
degree, become familiar with current techniques. Naur, in contrast, focuses on generally applicable theoretical content,
which does \emph{not} ascertain purpose or utility.

We see from the comparison that while the general approach of teaching datalogy hasn’t changed much over time, the
opposing party’s views have been altered dramatically. As such, discrepancies have been reduced over time.

\section{Concluding remarks}
In light of how datalogy has preserved its core values and curriculum virtually unchanged, and the fact that the ACM has
moved more towards this set of values, and changed their curriculum accordingly. We see that the gap between these has
been decreased over time, but a gap remains nevertheless. Namely, the ACM still tries to cover far more topics than what
Peter Naur believes to be essential.

\end{document}
